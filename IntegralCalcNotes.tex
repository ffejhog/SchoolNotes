% !TEX TS-program = pdflatex
% !TEX encoding = UTF-8 Unicode

% This is a simple template for a LaTeX document using the "article" class.
% See "book", "report", "letter" for other types of document.

\documentclass[11pt]{article} % use larger type; default would be 10pt

\usepackage[utf8]{inputenc} % set input encoding (not needed with XeLaTeX)

%%% Examples of Article customizations
% These packages are optional, depending whether you want the features they provide.
% See the LaTeX Companion or other references for full information.

%%% PAGE DIMENSIONS
\usepackage{geometry} % to change the page dimensions
\geometry{letterpaper} % or letterpaper (US) or a5paper or....
 \geometry{margin=1in} % for example, change the margins to 2 inches all round
% \geometry{landscape} % set up the page for landscape
%   read geometry.pdf for detailed page layout information

\usepackage{graphicx} % support the \includegraphics command and options

% \usepackage[parfill]{parskip} % Activate to begin paragraphs with an empty line rather than an indent

%%% PACKAGES
\usepackage{booktabs} % for much better looking tables
\usepackage{array} % for better arrays (eg matrices) in maths
\usepackage{paralist} % very flexible & customisable lists (eg. enumerate/itemize, etc.)
\usepackage{verbatim} % adds environment for commenting out blocks of text & for better verbatim
\usepackage{subfig} % make it possible to include more than one captioned figure/table in a single float
% These packages are all incorporated in the memoir class to one degree or another...

%%% HEADERS & FOOTERS
\usepackage{fancyhdr} % This should be set AFTER setting up the page geometry
\pagestyle{fancy} % options: empty , plain , fancy
\renewcommand{\headrulewidth}{0pt} % customise the layout...
\lhead{}\chead{}\rhead{}
\lfoot{}\cfoot{\thepage}\rfoot{}

%%% SECTION TITLE APPEARANCE
\usepackage{sectsty}
\allsectionsfont{\sffamily\mdseries\upshape} % (See the fntguide.pdf for font help)
% (This matches ConTeXt defaults)

%%% ToC (table of contents) APPEARANCE
\usepackage[nottoc,notlof,notlot]{tocbibind} % Put the bibliography in the ToC
\usepackage[titles,subfigure]{tocloft} % Alter the style of the Table of Contents
\renewcommand{\cftsecfont}{\rmfamily\mdseries\upshape}
\renewcommand{\cftsecpagefont}{\rmfamily\mdseries\upshape} % No bold!



%%% END Article customizations

%%% The "real" document content comes below...

\title{Brief Article}
\author{The Author}
%\date{} % Activate to display a given date or no date (if empty),
         % otherwise the current date is printed 
\everymath{\displaystyle}
\begin{document}
\section{Special Limits}
\begin{math}
\lim_{x \to 0} \frac{\sin x}{x}=1 \\
\\
\lim_{x \to 0} \frac{1- \cos x}{x}=0
\\
\end{math}
\section{Special Derivatives}
\begin{math}
\frac{\mathrm{d} [\sin x]}{\mathrm{d} x} = \cos x \\
\\
\frac{\mathrm{d} [\cos x]}{\mathrm{d} x} = - \sin x \\
\\
\frac{\mathrm{d} [\tan x]}{\mathrm{d} x} = \sec^{2} x \\
\\
\frac{\mathrm{d} [\sec x]}{\mathrm{d} x} = \sec x \tan x \\
\\
\frac{\mathrm{d} [\cot x]}{\mathrm{d} x} = - \csc^{2} x \\
\\
\frac{\mathrm{d} [\csc x]}{\mathrm{d} x} = - \csc x \cot x \\
\\
\frac{\mathrm{d} [arcsin x]}{\mathrm{d} x} =  \frac{1}{\sqrt{1-x^{2}}} \\
\\
\frac{\mathrm{d} [arccos x]}{\mathrm{d} x} = - \frac{1}{\sqrt{1-x^{2}}} \\
\\
\frac{\mathrm{d} [arctan x]}{\mathrm{d} x} =  \frac{1}{{1-x^{2}}} \\
\\
\frac{\mathrm{d} [arccot x]}{\mathrm{d} x} = - \frac{1}{{1-x^{2}}} \\
\\
\frac{\mathrm{d} [arcsec x]}{\mathrm{d} x} =  \frac{1}{\left | x \right |\sqrt{1-x^{2}}} \\
\\
\frac{\mathrm{d} [arccsc x]}{\mathrm{d} x} = - \frac{1}{\left | x \right |\sqrt{1-x^{2}}} \\
\\
\frac{\mathrm{d} [\ln x]}{\mathrm{d} x} = \frac{1}{x} \\
\\
\frac{\mathrm{d} [\log_{a} x]}{\mathrm{d} x} = \frac{1}{x\ln a} \\
\\
\frac{\mathrm{d} [e^{x}]}{\mathrm{d} x} = e^{x} \\
\\
\frac{\mathrm{d} [a^{x}]}{\mathrm{d} x} = a^{x}\ln a \\
\\
\end{math}

\section{Trig Rules}
\begin{math}
\int \tan x = \ln\left | \sec x \right | + c \\
\\
\int \sec x = \ln\left | \sec x + \tan x \right | + c \\
\\
\sin A \cos B = \frac{1}{2}[\sin(A-B) + \sin(A+B)] \\
\\
\sin A \sin B = \frac{1}{2}[\cos(A-B)-\cos(A+B)] \\
\\
\cos A \cos B = \frac{1}{2}[\cos(A-B)+\cos(A+B)] \\
\\
\sin^{2}x + \cos^{2}x = 1 \\
\\
\tan^{2}x +1 = \sec^{2}x \\
\\
\sin^{2}x = \frac{1}{2}(1-\cos 2x) \\
\\
\cos^{2}x = \frac{1}{2}(1+\cos 2x) \\
\\
\sin 2x = 2\sin x \cos x
\end{math}
\section{Trig Integrals}
\begin{math}
\int \sin^{m} x\cos^{n} x \\
\end{math}
1. If n is odd 
\begin{math} 
\rightarrow u=\sin x 
\end{math} \\
2. If m is odd
\begin{math} 
\rightarrow u=\cos x
\end{math} \\
3. If  n and m are even
\begin{math} 
\rightarrow 
\end{math}
Double Angle Formula \\
\\
\begin{math}
\int \tan^{m} x\sec^{n} x
\end{math} \\
1. If m is odd
\begin{math} 
\rightarrow u=\sec x
\end{math} \\
2. If n is even
\begin{math} 
\rightarrow u=\tan x
\end{math} \\
\section{Trig Substitution Rules}
\begin{math}
\int \sqrt {a^{2}-x^{2}} \rightarrow \sin \theta \\
\\
\int \sqrt {x^{2}-a^{2}} \rightarrow \sec \theta \\
\\
\int \sqrt {x^{2}+a^{2}} \rightarrow \tan \theta \\
\end{math}
\section{Improper Integrals}
\begin{math}
\int_{a}^{\infty } f(x) = \lim_{b \to \infty}\int_{a}^{b} f(x)
\end{math}
\section{Arc Length}
Cartisan \\
\\
\begin{math}
\int_{a}^{b} \sqrt{1+\left ( \frac{\mathrm{d} y}{\mathrm{d} x} \right )^{2}} \mathrm{d} x 
\end{math} \\
\\
\\
Parametric \\
\\
\begin{math}
\int_{a}^{b} \sqrt{\left ( \frac{\mathrm{d} x}{\mathrm{d} t} \right )^{2}+\left ( \frac{\mathrm{d} y}{\mathrm{d} t} \right )^{2}} \mathrm{d} t 
\end{math} \\
\\
\\
Polar \\
\\
\begin{math}
\int_{a}^{b} \sqrt{\left ( \frac{\mathrm{d} r}{\mathrm{d} \theta} \right )^{2}+r^{2}} \, \mathrm{d} \theta
\end{math} \\
\section{Area of Surface of Revolution}
\begin{math}
\int_{a}^{b}2\pi r \mathrm{ds}
\end{math} \\
\\
r 
\begin{math}
\rightarrow
\end{math}
Radius (Could be an equation or a single variable) \\
ds
\begin{math}
\rightarrow
\end{math}
The Arc Length function (changes depending on type of equation given, eg. Parametric, Polar(see Section 7))
\section{Parametric Equations}
Tangent horizontal when 
\begin{math}
\frac{\mathrm{d} y}{\mathrm{d} x} =0
\end{math} \\
Tangent vertical when 
\begin{math}
\frac{\mathrm{d} y}{\mathrm{d} x} = \mathrm{undef \, or} \, \infty
\end{math}
\\
\begin{math}
\mathrm{if} \, y=f(x)\, \mathrm{then} \left\{
\begin{array}{c l}     
    x=t\\
    y=f(x)
\end{array}\right. \\
\end{math}
\begin{math}
\frac{{y}'(t)}{{x}'(t)} = \frac{\frac{\mathrm{d} y}{\mathrm{d} t}}{\frac{\mathrm{d} x}{\mathrm{d} t}} \\
\\
\frac{\mathrm{d} ^{2} y}{\mathrm{d} x} = \frac{\mathrm{d} \left (\frac{\mathrm{d} y}{\mathrm{d} t}/{\frac{\mathrm{d} x}{\mathrm{d} t}} \right )}{\mathrm{d} x} \\
\end{math}
Area under parametric curve
\begin{math}
\int_{a}^{b} g(t){f}'(t) \mathrm{d}t
\end{math} \\
\section{Polar Equations}
Polar to Cartisan \\
\\
\begin{math}
\left\{
\begin{array}{c l}     
    x=r\cos \theta \\
    y=r\sin \theta
\end{array}\right. \\
\end{math} \\
\\
Cartisan to Polar \\
\\
\begin{math}
\left\{
\begin{array}{c l}     
    r^{2} = x^{2} + y^{2} \\
    \theta = tan^{-1}\left ( \frac{y}{x} \right )
\end{array}\right. \\
\end{math} \\
\\
Area of a polar curve \\
\\
\begin{math}
\int_{a}^{b} \frac{1}{2}[f(\theta)]^{2} \, \mathrm{d}\theta \\
\end{math}
\section{Series}
\subsection{P-Series}
\begin{math}
\sum_{n=1}^{\infty}\frac{1}{n^{p}} 
\left\{
\begin{array}{c l}     
    \, p>1 \,\,\, \mathrm{Converges} \\
    p\leq 1 \,\,\, \mathrm{Diverges}
\end{array}\right. \\
\end{math}
\subsection{Geometric Series}
\begin{math}
a,ar,ar^{2},... \sum_{n=1}^{\infty}ar^{n-1} 
\left\{
\begin{array}{c l}     
    -1<r<1 \,\,\, \mathrm{converges\, at} \,\, \frac{a}{1-r} \\
    \mathrm{otherwise} \,\,\, \mathrm{diverges} 
\end{array}\right. \\
\end{math} \\
\\
Partial sum for geometric series\\
\\
\begin{math}
s_{n}=\frac{a(1-r^{n})}{1-r}
\end{math}
\subsection{Limit Test}
\begin{math}
\mathrm{if}\, \lim_{n\to \infty}\neq 0 \,\, \mathrm{then} \, \sum_{n=1}^{\infty}a_{n} \, \mathrm{diverges} \\
\end{math}
\subsection{Integral Test}
\begin{math}
a_{n} = f(n)
\end{math}
\, where f is \\
1. Continuous \\
2. Positive \\
3. Decreasing \\
\begin{math}
\mathrm{Then} \, \sum_{n=1}^{\infty}a_{n} \, \mathrm{converges\, iff} \, \int_{1}^{\infty}f(x)\mathrm{d}x \, \mathrm{converges} \\
\end{math}
\subsection{Comparison Test}
Suppose
\begin{math}
a_{n}>b_{n}>0 \mathrm{\, for\, all\,} n\geq 1\\
\\
\mathrm{if} \, \sum_{n=1}^{\infty}a_{n} \, \mathrm{converges\, then} \, \sum_{n=1}^{\infty}b_{n} \, \mathrm{converges} \\
\mathrm{if} \, \sum_{n=1}^{\infty}b_{n} \, \mathrm{diverges\, then} \, \sum_{n=1}^{\infty}a_{n} \, \mathrm{diverges} \\
\end{math} \\
if 
\begin{math}
\lim_{n\to \infty} \frac{a_{n}}{b_{n}}= 
\end{math}
(a positive number)
then \,
\begin{math}
\sum_{n=1}^{\infty}a_{n} \, \mathrm{and} \, \sum_{n=1}^{\infty}b_{n} \,
\end{math}
have the same convergence/divergence
\subsection{Alternating Series Test}
\begin{math}
\sum_{n=1}^{\infty}(-1)^{n-1} a_{n} \ \mathrm{where} \ a_{n}>0 \\
\\
\mathrm{1. \ if} \ a_{n} \mathrm{is \ decreasing} \\
\mathrm{2. \ } \lim_{n\to \infty} a_{n}=0 \\
\\
\mathrm{then \ } \sum_{n=1}^{\infty}(-1)^{n-1} a_{n} \mathrm{ \ is \ convergent \ }
\end{math}
\subsection{Ratio Test/Root Test}
Let 
\begin{math}
L=\lim_{n\to \infty}\left | \frac{b_{n}+1}{b_{n}} \right | \\
\end{math}
\\
Let 
\begin{math}
L=\lim_{n\to \infty} \sqrt[n]{\left | b_{n} \right |}
\end{math} \\
if $L<1$ Series is absolutly convergent \\
if $L>1$(or $\infty$) Series is divergent \\
if $L=1$ Test is inconclusive
\subsection{Taylor and Maclaurin Series}
Taylor Series \\
\\
$f(x)=\sum_{n=0}^{\infty}\frac{f^{n\mathrm{(\leftarrow derivative)}}(a)}{n!}(x-a)^{n} \ \mathrm{for} \ \left | a-x \right | < R$ \\
\\
Maclaurin Series \\
\\
$f(x)=\sum_{n=0}^{\infty} \frac{f^{n\mathrm{(\leftarrow derivative)}}(0)}{n!}(x)^{n}$ \\
\section{Misc Rules}
\begin{math}
\int \frac{1}{a^{2}+x^{2}} = \frac{1}{a}\tan^{-1}\frac{x}{a} \\
\\
\tan^{-1}\infty = \frac{\pi}{2} \\
\\
\tan^{-1}-\infty = -\frac{\pi}{2} \\
\end{math}
\\
To find the intersection points between two curves, set them equal to each other \\
\\
Area between two curves is top curve minus bottom curve \\
\\
\end{document}
